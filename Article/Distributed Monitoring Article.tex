\documentclass[10pt,a4paper]{article}
\usepackage[utf8]{inputenc}
\usepackage[english]{babel}
\usepackage{amsmath}
\usepackage{amsfonts}
\usepackage{amssymb}
\usepackage{graphicx}
\usepackage{fullpage}
\usepackage{mathptmx}
\usepackage{setspace}
\usepackage[none]{hyphenat}
\usepackage{multicol}

%%%%%%%%%%%%%%%%% Macros %%%%%%%%%%%%%%%%%

\newcommand{\vecScheme}{\textit{Vector Scheme}}

%%%%%%%%%%%%%%%%% Document %%%%%%%%%%%%%%%%%

\begin{document}
\onehalfspacing
\sloppy
\setlength{\parindent}{0pt}


%%%%%%%%%%%%%%%%% Title %%%%%%%%%%%%%%%%%
\title{Bandwidth Efficient Distributed Monitoring Schemes}
\date{}
\maketitle


%%%%%%%%%%%%%%%%% Abstract %%%%%%%%%%%%%%%%%
\begin{multicols*}{2}
\begin{center}
\section*{Abstract}
\end{center}

%%%%%%%%%%%%%%%%% Introduction %%%%%%%%%%%%%%%%%
\section{Introduction}
Monitoring the a function over an aggregation of large amount of data which changes \\

Given convex f \label{fConvexity}

The monitoring objective is to determine whether:
\begin{equation}
\label{monitoringConstraint}
f(v) <= T
\end{equation}


%%%%%%%%%%%%%%%%% Previous Work %%%%%%%%%%%%%%%%%
\section{Previous Work}

%%%%%%%%%%%%%%%%% Vector Scheme %%%%%%%%%%%%%%%%%
\section{Vector Scheme}

The \vecScheme 's idea is to balance the data vectors of the servers. when a server's local data vector gets out of the function's bound, this scheme would like to balance it with other data vector. It would be done by incorporating \textit{slack vectors}, namely, $server_i$ would maintain a slack $\overrightarrow{s_i}$. It's important to note that the \vecScheme \  makes sure that (\label{vectorSchemSlackSumZero}\ref{vectorSchemSlackSumZero}) $sum{\overrightarrow{s_i}} = \textstyle \overrightarrow{0}$
In order to take into consideration these \textit{slacks}, a server raises a violation and initiates a communication channel with the coordinator if $f(v_i+s_i)$ exceeds the threshold; specifically, for a lower bound threshold, when $f(v_i+s_i) <= T$. This ensures that whenever all the local constraint hold, the global constraint mentioned in section [\ref{monitoringConstraint}]. proof due to (\ref{fConvexity}), (\ref{vectorSchemSlackSumZero}):
\begin{equation}
\label{vectorSchemeProof}
\begin{aligned}
 f(v)  \
	   ={} & f\left(\frac{1}{n} \sum\limits_{i=0}^{n}{v_i}\right)  \
        =   \frac{1}{n} f\left(\sum\limits_{i=0}^{n}{(v_i + s_i)}\right) \\
     & <=   \frac{1}{n} \sum\limits_{i=0}^{n}{f(v_i + s_i)}
       <=   \frac{1}{n}(n \cdot T)
       = T
\end{aligned}
\end{equation}

When a violation occurs, i.e. $f(v_i+s_i)>T$ at a certain server, (\ref{vectorSchemeProof}) cannot longer be proven so a \textit{violation resolution} has to occur. In the \textit{violation resolution} phase, the slack vectors are balanced so $f(v_i+s_i)$ would get inside the convex zone. When a server detects a local violation, it sends its local vector $(v_i + s_i)$ to the coordinator, which polls other servers for their local vector as well. When the average of those vectors is inside the convex zone, i.g. $f(E(v_i + s_i)) <= T$. after that, the coordinator sends the average vector (k - number of polled nodes plus violated node) -- $\frac{1}{k}\sum{(v_i + s_i)}$ to the polled nodes as well as the violated node, which update their slack to be $s_i \leftarrow -v_i + \frac{1}{k}\sum{(v_i + s_i)} $. Note that condition (\ref{vectorSchemSlackSumZero}) still holds. \\
When all the nodes are polled and the average vector still isn't inside the convex zone, a \textit{full sync} has to be done, the real value of $f(v)$ is known, so the upper bound and lower bound reset to $(1 \pm \varepsilon )f(v)$ and the monitoring continues.

%%%%%%%%%%%%%%%%% Value Scheme %%%%%%%%%%%%%%%%%
\section{Value Scheme}

%%%%%%%%%%%%%%%%% Distance Scheme %%%%%%%%%%%%%%%%%
\section{Distance Scheme}

\subsection{Distance Lemma}

%%%%%%%%%%%%%%%%% Data Resolution %%%%%%%%%%%%%%%%%
\section{Sketched Data Resolution}

%%%%%%%%%%%%%%%%% Data Resolution %%%%%%%%%%%%%%%%%
\section{Sketched Change Resolution}

%%%%%%%%%%%%%%%%% Experimental Results %%%%%%%%%%%%%%%%%
\section{Experimental Results}

\ 

%%%%%%%%%%%%%%%%% References %%%%%%%%%%%%%%%%%
\begin{center}
\section*{References}
\end{center}

\end{multicols*}
\end{document}
